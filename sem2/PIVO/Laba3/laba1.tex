\documentclass[10pt, a4paper]{article}
\usepackage{CJKutf8}
\usepackage[russian, english]{babel}
\newcommand{\RomanNumeralCaps}[1]
    {\MakeUppercase{\romannumeral #1}}
\usepackage[utf8]{inputenc}
\usepackage{multicol} % Колонки
\usepackage{setspace} % Межстрочный интервал
\usepackage{ragged2e} % Выравнивания текста по ширине в документе.
\usepackage{fancyhdr} % Настройки верхнего и нижнего колонтитулов в документе.
\usepackage{titlesec} % Стилей заголовков разделов в документе.
\usepackage{enumitem} % Настройки списков в документе.
\usepackage{graphicx}
\usepackage{float} % "Плавающие" картинки
\usepackage{wrapfig} % Обтекание фигур (таблиц, картинок и прочего)
\usepackage[left=1.9cm,right=1.9cm, top=2.2cm,bottom=2.5cm]{geometry}

\justifying % Выравнивает текст по ширине.
\fancyhf{} % Очищает все верхние и нижние колонтитулы.
\renewcommand{\headrulewidth}{0pt} % remove the header rule
\cfoot{\vskip -1.5cm \thepage} % Устанавливает номер страницы в нижнем колонтитуле.

\linespread{0.8} % Устанавливает межстрочный интервал в 0.84.


\setlength{\columnsep}{0.5cm}
\setcounter{page}{162}
\renewcommand{\thesection}{\Roman{section}} % Устанавливает стиль нумерации разделов в виде заглавных римских цифр.

\titleformat{\section}{\footnotesize\centering\sc}{\thesection.}{0cm}{}[] % Настраивает стиль заголовков разделов.

\begin{document}

\selectlanguage{english}
\fontsize{10}{14}\selectfont
\begin{multicols}{2}
\setlength{\parindent}{0.8cm}
\par
\setlength{\parindent}{0.3cm}
\fontsize{10}{15}\selectfont
\begin{center}
    \RomanNumeralCaps{5}. The subject area and ontology 
    \\
    of GI disease
\end{center}
\par
Gastrointestinal (GI) diseases are one of the most common problems in medical practice worldwide. They cover a wide range of conditions, from functional disorders to serious pathologies such as peptic ulcers and cancer. According to the World Health Organization (WHO), GI diseases are the leading causes of death and disability worldwide.
\par
GI disease statistics:
\setlength{\parindent}{0.0cm}
\par
\setlength{\parindent}{0.0cm}
\fontsize{10}{15}\selectfont
\begin{itemize}
    \item according to the WHO, in 2020, GI diseases are the cause of death for more than 4 million people worldwide;
    \item according to studies conducted in different countries, GI diseases account for up to 25\% of all reasons for visits to general practitioners;
    \item some of the most common GI diseases include peptic ulcer disease, gastric and duodenal ulcers, gastritis, colitis, irritable bowel syndrome (IBS), gallstones, pancreatitis, and GI cancer;
\end{itemize}
\fontsize{8}{8}\selectfont
\begin{minipage}{0.45\textwidth} % 45% ширины страницы для первой картинки
    \includegraphics[width=0.99\textwidth]{third.jpg}
    \caption{A fragment of the ontology of a medical record that allows you to store clarifying information}
    \label{ris:third}
\end{minipage}

\setlength{\parindent}{0.8cm}
\par
\setlength{\parindent}{0.3cm}
\fontsize{10}{15.93}\selectfont
The International Classification of Diseases, 10th Revision (ICD-10) provides a coding system for diseases
used in medical statistics and diagnosis. GI diseases
are described in ICD-10 section K00-K93. This section includes a wide range of conditions, from dental
problems to diseases of the liver, pancreas, and other
GI organs. Diseases of this area include functional disorders, inflammatory processes, infections, tumors, and
other pathologies specific to the GI tract. They can be
manifested by various symptoms such as abdominal pain,
diarrhea, constipation, nausea, vomiting and others. The
definition and classification of GI diseases according to
ICD-10 is important for statistical analysis, morbidity
studies and health care planning.
\par 
Fig. ~\ref{ris:third} shows the formalization of the digestive organs domain using OSTIS technology. This formalization
includes the development of an appropriate ontology
structuring information about GI diseases according to
the main sections of the International Classification of
Diseases 10th Revision (ICD-10) [12].
\par
The first section of the digestive organ ontology covers
the anatomical structure and functions of organs including stomach, liver, pancreas, intestine and others. Each
organ is presented as a separate entity described by its
anatomical features and functions. The subject matter is
further divided into various sections, including functional
disorders, infections, tumors and other pathologies, in
accordance with ICD-10. Each section contains the relevant classes of diseases and their associated medical
conditions, symptoms and treatments.
\par
In the context of the study of the subject area of
digestive organs, special attention is paid to the stomach, considered on the example of gastritis in its usual
and hyperacidic forms. Each disease corresponds to a
reference marker set by the expert, which can be tissue
or drug-specific. In addition, each disease has etiologic
markers, which are multiple indicators that point to
possible sources of the disease, such as bacteria, viruses,
and other factors.
\par
Organs in the digestive system can be in three states:
disease state (more than 80\% similarity), risk state (50%
to 80\% similarity), and non-risk state (healthy organ, less
than 50\% similarity). This approach allows the system to
classify organs according to their current status based on
analysis of user data.
\par
The formalization of the ontology fragment and its
corresponding knowledge base, presented in the figure,
allows not only to treat diseases after their manifestation,
but also to carry out the tasks of early diagnosis and
prevention of the disease at early stages. This methodology allows integrating reference and etiological markers
of diseases into the knowledge base, which provides
the system with access to information for analyzing and
processing medical indicators at a deeper level, which is
discussed in the works of Rostovtsev V. N. [13]–[15].

\clearpage
\end{multicols}

%\begin{figure}[h] % 'h' - здесь
    %\centering
    %\includegraphics[width=0.88\textwidth]{figure 4.jpg}
    %\caption{Fragment of medical record ontology}
    %\label{fig:fourth}
%\end{figure}

\begin{figure}[h]
    \center{\includegraphics[width=0.88\linewidth]{fourth.jpg}}
    \caption{Fragment of medical record ontology}
    \label{ris:fourth}
\end{figure}

\selectlanguage{english}
\fontsize{10}{14}\selectfont
\begin{multicols}{2}
\setlength{\parindent}{0.8cm}
\par
\setlength{\parindent}{0.3cm}
\fontsize{10}{15}\selectfont
\begin{center}
    \RomanNumeralCaps{6}. Conclusion 
\end{center}
\par
The integration of Open Semantic Technology for
Intelligent Systems (OSTIS) into medical information
systems presents a promising solution to the challenge
of data format incompatibility. OSTIS offers innovative
tools and approaches for creating semantically compat-

\columnbreak % Разделить колонки

\noindent ible medical systems capable of efficiently processing
and storing data regardless of their original format and
structure.
\par
One of the key features of OSTIS is its ability to
unify various types of knowledge into a single database.
This centralized approach allows for the organization and

\end{multicols}

\selectlanguage{english}
\fontsize{10}{14}\selectfont
\begin{multicols}{2}
\setlength{\parindent}{0.8cm}
\par
\setlength{\parindent}{0.3cm}
\fontsize{10}{15}\selectfont

\noindent structuring of medical data according to unified semantic
standards, ensuring high compatibility and interoperability.
\par
Furthermore, the flexibility and adaptability of OSTIS
enable the customization of systems to meet the specific
requirements and standards of each country, including
Belarus, Russia, and Kazakhstan. This adaptability
facilitates seamless integration into existing healthcare
infrastructures.
\par
The automatic conversion and matching of data in
different formats represent a significant advantage of OSTIS.
This capability eliminates compatibility issues and
facilitates smooth information exchange between various
medical systems and institutions, ultimately enhancing
system efficiency and the quality of healthcare delivery.
\par
In summary, the application of OSTIS technology
offers an effective and promising approach to addressing
data format incompatibility in medical information
systems. It fosters the creation of modern and innovative
healthcare systems capable of adapting to diverse requirements
and changes in the medical field, which is crucial
for improving the quality and accessibility of healthcare
in different countries.
\par
\setlength{\parindent}{0.0cm}
\par
\setlength{\parindent}{0.0cm}
\fontsize{7}{7}\selectfont
    \begin{thebibliography}{15}
        \bibitem[1]{} Zhan Y. et al. Investigating the role of Cybersecurity’s perceived
        threats in the adoption of health information systems. \textit{Heliyon},
        2024, Vol. 10, № 1.
        \bibitem[2]{} Yang P. et al. LMKG: A large-scale and multi-source medical knowledge graph for intelligent medicine applications. \textit{Knowledge-Based Systems}, 2024, Vol. 284, P. 111323.
        \bibitem[3]{} Yakimov, D. A., Vygovskaya, N. V., Drozdov, I. V. Development
        of a Medical Information System with Data Storage and Intelligent Image Analysis. \textit{Digital Transformation}, 2024, Vol. 30, № 1,
        pp. 71–80.
        \bibitem[4]{} Troqe B., Holmberg G., Lakemond N. Making decisions with AI in complex intelligent systems. \textit{s. Research Handbook on Artificial
        Intelligence and Decision Making in Organizations}. Edward Elgar Publishing, 2024, pp. 160–178. [5] Shawwa L. The use of telemedicine
        \bibitem[5]{} Shawwa L. The use of telemedicine in medical education and patient care. \textit{Cureus}, 2023, Vol. 15, №. 4.
        \bibitem[6]{} Chauhan P. et al Breaking Barriers for Accessible Health Programs: The Role of Telemedicine in a Global Healthcare Transformation. \textit{Transformative Approaches to Patient Literacy and Healthcare Innovation}. IGI Global, 2024, pp. 283–307.
        \bibitem[7]{} Kontseptsiya razvitiya elektronnogo zdravookhraneniya Respubliki Belarus’ na period do 2022 goda: prikaz Ministerstva zdravookhraneniya Respubliki Belarus’ ot 20 marta 2018 g. [Concept for the development of electronic health care of the Republic of Belarus for the period until 2022: order of the Ministry of Health of the Republic of Belarus dated March 20, 2018]. Ministerstvo zdravookhraneniya Respubliki Belarus’ [Ministry of Health of the Republic of Belarus], 2018, № 244
        \bibitem[8]{} Ob utverzhdenii form pervichnoi meditsinskoi dokumentatsii v ambulatorno-poliklinicheskikh organizatsiyakh: prikaz Ministerstva zdravookhraneniya Respubliki Belarus’ ot 30 avgusta 2007 g. [On approval of forms of primary medical documentation in outpatient clinics: order of the Ministry of Health of the Republic of Belarus dated August 30, 2007.]. Ministerstvo zdravookhraneniya Respubliki Belarus’ [Ministry of Health of the Republic of Belarus], 2007, № 710
        \bibitem[9]{} Ob utverzhdenii unifitsirovannykh form meditsinskoi dokumentatsii, ispol’zuemykh v meditsinskikh organizatsiyakh, okazyvayushchikh meditsinskuyu pomoshch’ v ambulatornykh usloviyakh, i poryadkov po ikh zapolneniyu: prikaz Ministerstva 
        \\
        zdravookhraneniya Rossiiskoi Federatsii ot 15 dekabrya 2014 g. [On approval of unified forms of medical documentation used in medical organizations providing medical care in outpatient settings, and procedures for filling them out: order of the Ministry of Health of the Russian Federation dated December 15, 2014.]. Ministerstvo zdravookhraneniya Respubliki Belarus’ [Ministry of Health of the Republic of Belarus], 2014, № 834n
        \bibitem[10]{} Ob utverzhdenii form uchetnoi dokumentatsii v oblasti
        zdravookhraneniya: prikaz i.o. Ministra zdravookhraneniya
        Respubliki Kazakhstan ot 30 oktyabrya 2020 g. [On approval of
        forms of accounting documentation in the field of healthcare:
        order of acting. Minister of Health of the Republic of
        Kazakhstan dated October 30, 2020]. Ministerstvo yustitsii
        Respubliki Kazakhstan [Ministry of Justice of the Republic of
        Kazakhstan], 2020, № KR DSM-175/2020.
        \bibitem[11]{} O poryadke funktsionirovaniya i ispol’zovaniya tsentralizovannoi
        informatsionnoi sistemy zdravookhraneniya: postanovlenie Soveta
        Ministrov Respubliki Belarus’, 13 maya 2021 g. [On the procedure
        for the functioning and use of a centralized health information
        system: Resolution of the Council of Ministers of the Republic
        of Belarus, May 13, 2021]. Natsional’nyi pravovoi Internetportal Respublsiki Belarus’ [National legal Internet portal of the
        Republic of Belarus], 2021, № 267, 15.05.2021, 5/49050.
        \bibitem[12]{}  International Statistical Classification of Diseases and Related
        Health Problems (ICD-10): Official Version. Tenth revised edition. Geneva: World Health Organization, 1992.
        \bibitem[13]{}  Rostovtsev, V. Intelligent health monitoring systems. Open semantic technologies for intelligent systems, 2023, Iss. 7, pp. 237–240.
        \bibitem[14]{} Rostovtsev, V. N., Kobrinskii, B. A. Principles and possible ways
        of building an intelligent system of integral medicine. Open
        semantic technologies for intelligent systems, 2021, Iss. 5, pp.
        225–228.
        \bibitem[15]{}  Rostovtsev, V. N., Rodionova, O. S. Principles of ostis-system
        of automatic diagnosis design. Open semantic technologies for
        intelligent systems, 2018, pp.341–346
    \end{thebibliography}

\setlength{\parindent}{0.8cm}
\par
\setlength{\parindent}{0.3cm}
\fontsize{10}{15}\selectfont
\selectlanguage{russian}
\begin{center}
\fontsize{9}{13}\selectfont
\textbf{ИНТЕГРАЦИЯ И СТАНДАРТИЗАЦИЯ В
\\
ИНТЕЛЛЕКТУАЛЬНЫХ МЕДИЦИНСКИХ
\\
СИСТЕМАХ НОВОГО ПОКОЛЕНИЯ НА
\\
ОСНОВЕ ТЕХНОЛОГИИ OSTIS}
\end{center}
\setlength{\parindent}{0.8cm}
\par
\setlength{\parindent}{0.3cm}
\fontsize{9}{15}\selectfont
\selectlanguage{russian}
\par
    Крищенович В. А., Сальников Д. А., Захарьев В. А.
\setlength{\parindent}{0.8cm}
\setlength{\parindent}{0.3cm}
\fontsize{9}{15}\selectfont
\selectlanguage{russian}
\par
В статье рассматривается интеграция международ-
ных медицинских стандартов в России, Беларуси и
Казахстане с применением семантических технологий.
Предлагается подход к интеграции и стандартизации
медицинских данных на основе применения техноло-
гии OSTIS. Приводится пример разработки фрагмента
онтологии на основе различных стандартов медицин-
ских карт в интеллектуальных медицинских системах.
Преимуществами такой интеграции являются улуч-
шение обмена медицинской информацией, упрощение
процесса диагностики и лечения, а также возможность
создания единого медицинского пространства в рам-
ках региона.
\par
\hfill Received 25.03.2024
\end{multicols}

\end{document}
